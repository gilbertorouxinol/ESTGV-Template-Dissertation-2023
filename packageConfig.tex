%Polytechnic Institute of Viseu
%School of Technology and Management of Viseu
%File Template Dissertation Created by Gilberto Rouxinol on 2023
%https://www.overleaf.com/
%Copyright © https://ctan.org
%Copyright © MIT License 2023 Gilberto Rouxinol


%Texto
\usepackage[utf8]{inputenc}
\usepackage[portuguese]{babel}

%Texto cor
\usepackage{xcolor}

%Bibliografia
\usepackage[backend=biber, style=numeric, sorting=ynt]{biblatex}
\addbibresource{G_RefBiblio.bib}
\usepackage{csquotes}

%Configuração da página
\usepackage[a4paper,left=3.0cm,right=3.0cm,top=2.5cm,bottom=2.5cm]{geometry}

%Alinhar equações
\usepackage{amsmath}

%Fontes ou símbolos matemáticos
\usepackage{amssymb}

%Referências de figuras
\usepackage{graphicx}

%Colocar subfiguras
\usepackage{caption}
\usepackage{subcaption}

%Escrever unidades do SI
\usepackage{siunitx}

%Acrescentar pdfs
\usepackage{pdfpages}

%Paginação
\usepackage{fancyhdr}
\pagestyle{fancy}
\fancyhf{}
\renewcommand{\headrulewidth}{0pt}
\fancyfoot[R]{\thepage}

%Criar os níveis de profundidade de subsubsubsection e de subsubsubsubsection
\setcounter{secnumdepth}{5}
\makeatletter
\newcommand\subsubsubsection{\@startsection{paragraph}{4}{\z@}{-2.5ex\@plus -1ex \@minus -.25ex}{1.25ex \@plus .25ex}{\normalfont\normalsize\bfseries}}
\newcommand\subsubsubsubsection{\@startsection{subparagraph}{5}{\z@}{-2.5ex\@plus -1ex \@minus -.25ex}{1.25ex \@plus .25ex}{\normalfont\normalsize\bfseries}}
\makeatother

%Indicar o nível de profundidade do índice geral
\setcounter{tocdepth}{5}

%Renomear o nome por defeito de secções padrão
\addto\captionsportuguese{\renewcommand{\contentsname}{ÍNDICE GERAL}}
\addto\captionsportuguese{\renewcommand{\listfigurename}{ÍNDICE DE FIGURAS}}
\addto\captionsportuguese{\renewcommand{\listtablename}{ÍNDICE DE TABELAS}}

%Contemplar o nome de secções não padrão no índice geral
\addcontentsline{toc}{section}{ÍNDICE DE TABELAS}
\addcontentsline{toc}{section}{ÍNDICE DE FIGURAS}
\addcontentsline{toc}{section}{LISTA DE SIGLAS/ABREVIATURAS}
\addcontentsline{toc}{section}{LISTA DE SÍMBOLOS}

%Colocar a numeração da secção à esquerda da numeração das equações, tabelas e figuras
\numberwithin{equation}{section}
\numberwithin{table}{section}
\numberwithin{figure}{section}
