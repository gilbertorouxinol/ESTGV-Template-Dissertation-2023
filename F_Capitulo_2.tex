Nulla ac nisl. Nullam urna nulla, ullamcorper in, interdum sit amet, gra- vida ut, risus. Aenean ac enim. In luctus. Phasellus eu quam vitae turpis viverra pellentesque. Duis feugiat felis ut enim. Phasellus pharetra, sem id porttitor sodales, magna nunc aliquet nibh, nec blandit nisl mauris at pede. Suspendisse risus risus, lobortis eget, semper at, imperdiet sit amet, quam. Quisque scelerisque dapibus nibh. Nam enim. Lorem ip- sum dolor sit amet, consectetuer adipiscing elit. Nunc ut metus. Ut metus justo, auctor at, ultrices eu, sagittis ut, purus. Aliquam aliquam.

\subsection{Nome da subsecção 1 do capítulo 2}
Duis aliquet dui in est.  \textcolor{red}{referência figura} \ref{fig:3juntas} Donec eget est. Nunc lectus odio, varius at, fermentum in, accumsan non, enim. Aliquam erat volutpat. Proin sit amet nulla ut eros consectetuer cursus. Phasellus dapibus aliquam justo. Nunc laoreet. Donec consequat placerat magna. \textcolor{red}{referência figura} \ref{fig:juntos2}.

\begin{figure}[!htbp]
    \centering
    \begin{subfigure}{0.3\textwidth}
        \centering
        \includegraphics[width=\textwidth]{Figuras/tablet1-2022.jpg}
        \caption{}
        \label{fig:juntos1}
    \end{subfigure}
    \hfill
    \begin{subfigure}{0.3\textwidth}
        \centering
        \includegraphics[width=\textwidth]{Figuras/tablet1-2022.jpg}
        \caption{}
        \label{fig:juntos2}
    \end{subfigure}
    \hfill
    \begin{subfigure}{0.3\textwidth}
        \centering
        \includegraphics[width=\textwidth]{Figuras/tablet1-2022.jpg}
        \caption{}
        \label{fig:juntos3}
    \end{subfigure}
    \caption{Nome geral da figura: (a) nome da figura a; (b) nome da figura b; (c) nome da figura c}
    \label{fig:3juntas}
\end{figure}

Duis pretium tincidunt justo. Sed sollicitudin vestibulum quam. Nam quis ligula. Vivamus at metus. Etiam imperdiet imperdiet pede. Aenean turpis. Fusce augue velit, scelerisque sollicitudin, dictum vitae, tempor et, pede. Donec wisi sapien, feugiat in, fermentum ut, sollicitudin adipiscing, metus.

\subsubsection{Nome da subsubsecção 1 da subsecção 1}
Blat \textcolor{red}{referência equação} \ref{eq:funcao} sed eleifend, eros sit amet faucibus elementum, urna sapien consectetuer mauris, quis egestas leo justo non risus. Morbi non felis ac libero vul- putate fringilla. Mauris libero eros, lacinia non, sodales quis, dapibus porttitor, pede. 

\begin{equation}
    f(x) =\left\{ 
    \begin{aligned}
          &0, && \forall x\in(-\infty,0] \\
          & e^{-\frac{1}{x^2}}, && \forall x\in[0,\infty) 
    \end{aligned}
    \label{eq:funcao}
    \right.
\end{equation}

Class aptent taciti sociosqu ad litora torquent per conu- bia nostra, per inceptos hymenaeos. Morbi dapibus mauris condimentum nulla. Cum sociis natoque penatibus et magnis dis parturient montes, nascetur ridiculus mus. Etiam sit amet erat. Nulla varius. Etiam tinci- dunt dui vitae turpis. Donec leo. Morbi vulputate convallis est. Integer aliquet. Pellentesque aliquet sodales urna.

\subsubsection{Nome da subsubsecção 2 da subsecção 1}
Class aptent taciti sociosqu ad litora torquent per conubia nostra, per inceptos hymenaeos. Aenean nonummy turpis id odio. Integer euismod imperdiet turpis. Ut nec leo nec diam imperdiet lacinia. Etiam eget lacus eget mi ultricies posuere. In placerat tristique tortor. Sed porta vestibulum metus. Nulla iaculis sollicitudin pede. Fusce luctus tellus in dolor. Curabitur auctor velit a sem. Morbi sapien. Class aptent taciti sociosqu ad litora torquent per conubia nostra, per inceptos hymenaeos. Donec adipiscing urna vehicula nunc. Sed ornare leo in leo. In rhoncus leo ut dui. Aenean dolor quam, volutpat nec, fringilla id, consectetuer vel, pede.

\subsection{Nome da subsecção 2 do capítulo 2}
Class aptent taciti sociosqu ad litora torquent per conubia nostra, per inceptos hymenaeos. Aenean nonummy turpis id odio. Integer euismod imperdiet turpis. Ut nec leo nec diam imperdiet lacinia. Etiam eget lacus eget mi ultricies posuere. In placerat tristique tortor. Sed porta vestibulum metus \textcolor{red}{referência tabela} \ref{tab:tabMulti}

\begin{table}[ht]
    \centering
    \caption{Nome da tabela}
    \begin{tabular}{cccc}
        \hline
        \multicolumn{2}{c}{Lado A} & \multicolumn{2}{c}{Lado B}\\
        Texto A & $\alpha$ & Texto B & $\beta$ \\
        \hline
       1  & 100 & 5 & 500 \\
       2  & 200 & 6 & 600 \\
       3  & 300 & 7 & 700 \\
       4  & 400 & 8 & 800 \\
        \hline
    \end{tabular}
    \label{tab:tabMulti}
\end{table}

Nulla iaculis sollicitudin pede. Fusce luctus tellus in dolor. Curabitur auctor velit a sem. Morbi sapien. Class aptent taciti sociosqu ad litora torquent per conubia nostra, per inceptos hymenaeos. Donec adipiscing urna vehicula nunc.

\subsection{Nome da subsecção 3 do capítulo 2}
Fusce suscipit cursus sem. Vivamus risus mi, egestas ac, imperdiet va- rius, faucibus quis, leo. Aenean tincidunt. Donec suscipit. Cras id justo quis nibh scelerisque dignissim. Aliquam sagittis elementum dolor. Aenean consectetuer justo in pede. Curabitur ullamcorper ligula nec orci. Aliquam purus turpis, aliquam id, ornare vitae, porttitor non, wisi. Maecenas luctus porta lorem. Donec vitae ligula eu ante pretium varius.

\subsubsection{Nome da subsubsecção 1 da subsecção 3}
Praesent sed neque id pede mollis rutrum. Vestibulum iaculis risus. \textcolor{red}{referencia tabela} \ref{tab:tabAB} Pellentesque lacus. Ut quis nunc sed odio malesuada egestas. Duis a ma- gna sit amet ligula tristique pretium. Ut pharetra. Vestibulum imperdiet magna nec wisi. Mauris convallis. Sed accumsan sollicitudin massa \textcolor{red}{referencia tabela} \ref{tab:tabA}

\begin{table}[h]
    \centering
    \caption{Nome geral das tabelas: (a) nome tabela A; (b) nome tabela B}
    \begin{subtable}[h]{0.4\textwidth}
        \centering
        \caption{}
        \begin{tabular}{llll}
            \hline
              & A & B & C \\
            \hline
           X  & 1 & 2 & 3 \\
           Y  & 4 & 5 & 6 \\
           Z  & 7 & 8 & 9 \\
            \hline
        \end{tabular}
        \label{tab:tabA}
    \end{subtable}
    \begin{subtable}[h]{0.4\textwidth}
        \centering
        \caption{}
        \begin{tabular}{llll}
            \hline
              & A & B & C \\
            \hline
           X  & 1 & 2 & 3 \\
           Y  & 4 & 5 & 6 \\
           Z  & 7 & 8 & 9 \\
            \hline
        \end{tabular}
        \label{tab:tabB}
    \end{subtable}
    \label{tab:tabAB}
\end{table}

Sed id enim. Nunc pede enim, lacinia ut, pulvinar quis, suscipit semper, elit. Cras accumsan erat vitae enim. Cras sollicitudin. Vestibulum rutrum blandit massa.

\subsubsection{Nome da subsubsecção 2 da subsecção 3}
Sed gravida lectus ut purus. Morbi laoreet magna. Pellentesque eu wisi. Proin turpis. Integer \textcolor{red}{referência equação} \ref{eq:def} sollicitudin augue nec dui. Fusce lectus. Vivamus faucibus nulla nec lacus. Integer diam. 

\begin{equation}
    \label{eq:def}
    [\Omega] = 
    \left[
    \begin{array}{ccc}
        x_{11} & x_{12} & x_{13} \\
        x_{21} & x_{22} & x_{23} \\
        x_{31} & x_{32} & x_{33} \\
    \end{array}
    \right]
\end{equation}

Pellentesque sodales, enim feugiat cursus volutpat, sem mauris dignissim mauris, quis conse- quat sem est fermentum ligula. Nullam justo lectus, condimentum sit amet, posuere a, fringilla mollis, felis. Morbi nulla nibh, pellentesque at, nonummy eu, sollicitudin nec, ipsum. Cras neque. Nunc augue. Nullam vitae quam id quam pulvinar blandit. Nunc sit amet orci. Aliquam erat elit, pharetra nec, aliquet a, gravida in, mi. Quisque urna enim, viverra quis, suscipit quis, tincidunt ut, sapien. Cras placerat consequat sem. Curabitur ac diam. Curabitur diam tortor, mollis et, viverra ac, tempus vel, metus.
